%!TEX root = ../thesis.tex
% First page
\thispagestyle{empty}
\vspace*{30mm}\noindent
\begin{center}
{\LARGE\sf\maintitle}\\[7mm]
{\Large\sf\subtitle}\\[4.5cm] %\\[7mm]
{\Large\sf \@author}
\end{center}

\newpage
\thispagestyle{empty}

% Page with c logo etc.
\vspace*{\fill}

\small\noindent The work described in this thesis was carried out at the Coherence and Quantum Technology group of the Department of Applied Physics and Science Education at Eindhoven University of
Technology. 

\vspace{\baselineskip}

\noindent\bgroup\small
A catalogue record is available from the Eindhoven University of Technology Library.\\
ISBN: \isbn\\[4mm]
Typeset by the author using the pdf \LaTeX \ documentation system.\\
Cover design: Name of cover designer \\
Reproduction: \printer\\[8mm]
\copyright 2026 by \@author. All rights reserved.
\egroup

\newpage
\thispagestyle{empty}


% Title page

\vspace*{30mm}
\begin{center}
{\LARGE\maintitle}\\[7mm]
{\Large\subtitle}\\[30mm] %\\[7mm]
{\large\textsc{Proefschrift}}\\[8mm]
ter verkrijging van de graad van doctor aan de\\
Technische Universiteit Eindhoven, op gezag van de\\
rector magnificus \rector, voor een\\
commissie aangewezen door het College voor\\
Promoties, in het openbaar te verdedigen\\
op \defensedate\ om \defensetime\ uur\\[8mm]
door\\[8mm]
\@author\\[8mm]
geboren te \placeofbirth
\end{center}
\vfill

\newpage
\thispagestyle{empty}

\noindent Dit proefschrift is goedgekeurd door de promotoren en de samenstelling van de promotiecommissie is als volgt:

\vspace{\baselineskip}

\noindent
\begin{tabular}{@{}l p{9.8cm}}
voorzitter:                 &   prof. dr. xx. xxxxxxxxx \\ \\
promotor:                   &   prof. dr. xx. xxxxxxxxx \\
co-promotor:                &   prof. dr. xx. xxxxxxxxx \\
leden:                      &   prof. dr. xx. xxxxxxxxx \\
                            &   prof. dr. xx. xxxxxxxxx \\
                            &   dr. xx. xxxxxxxxx \\
                            &   dr. xx. xxxxxxxxx \\
\end{tabular}

\vfill
\noindent
Het onderzoek dat in dit proefschrift wordt beschreven is uitgevoerd in overeenstemming met de TU/e Gedragscode Wetenschapsbeoefening.
